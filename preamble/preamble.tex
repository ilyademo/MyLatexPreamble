\documentclass[a4paper, 14pt]{article}

\usepackage[english, russian]{babel} 
\usepackage{extsizes}
% Поддержка языков
\usepackage[T2A]{fontenc}
\usepackage[utf8]{inputenc}
\usepackage{hyphenat}
% Настройка отступов от краев страницы
\usepackage[left=2cm, right=2cm, top=2cm, bottom=2cm]{geometry}
\usepackage{titlesec}
%\titleformat{\section}{\filcenter\normalfont\Large\bfseries}{\thesection.}{0.2em}{}
%\titleformat{\subsection}{\filcenter\normalfont\large\bfseries}{\thesubsection.}{1em}{}
\usepackage{pgfplots}
%\usepackage[font=normalsize, labelformat=parens, labelsep=gobble]{caption}
\pgfplotsset{compat=1.9}
\usepackage{titleps} % Колонтитулы
\usepackage{subfig} % Для подписей к рисункам и таблицам
\usepackage{graphicx} % для вставки картинок
\graphicspath{{./img/}} % Путь до папки с изображениями
% Пакет для отрисовки графиков
\usepackage{tikz}
\usetikzlibrary{arrows,positioning,shadows}
\usepackage{stmaryrd} % Стрелки в формулах
\usepackage{indentfirst} % Красная строка после заголовка
\usepackage{hhline} % Улучшенные горизонтальные линии в таблицах
\usepackage{multirow} % Ячейки в несколько строчек в таблицах
\usepackage{longtable} % Многостраничные таблицы
\usepackage{paralist,array} % Список внутри таблицы
\usepackage[normalem]{ulem}  % Зачеркнутый текст
\usepackage{upgreek, tipa} % Красивые греческие буквы
\usepackage{amsmath, amsfonts, amssymb, amsthm, mathtools} % ams пакеты для математики, табуляции
\usepackage{nicematrix} % Особые матрицы pNiceArray
\usepackage{ifthen}
\usepackage{float}
\usepackage{lipsum} 

%\numberwithin{equation}{subsection}

\counterwithin{figure}{section}
\counterwithin{table}{section}

\let\oldsection\section% Store \section
\renewcommand{\section}{% Update \section
	\renewcommand{\theequation}{\thesection.\arabic{equation}}% Update equation number
	\oldsection}% Regular \section
\let\oldsubsection\subsection% Store \subsection
\renewcommand{\subsection}{% Update \subsection
	\renewcommand{\theequation}{\thesubsection.\arabic{equation}}% Update equation number
	\oldsubsection}% Regular \subsection

\linespread{1.25} % Межстрочный интервал
\setlength{\parindent}{1.25cm} % Табуляция
\setlength{\parskip}{0cm}

% Пакет для красивого выделения кода%
\usepackage{minted}
\setminted{fontsize=\footnotesize}
\definecolor{bg}{rgb}{0.95,0.95,0.92}
\newminted[code]{C++}{breaklines, bgcolor=bg, linenos}

% Добавляем гипертекстовое оглавление в PDF
\usepackage[
bookmarks=true, colorlinks=true, unicode=true,
urlcolor=black,linkcolor=black, anchorcolor=black,
citecolor=black, menucolor=black, filecolor=black,
]{hyperref}

% Использование двух колонок
\usepackage{multicol}

% Убрать переносы слов
\tolerance=1
\emergencystretch=\maxdimen
\hyphenpenalty=10000
\hbadness=10000

\newpagestyle{main}{
	% Верхний колонтитул
	\setheadrule{0cm} % Размер линии отделяющей колонтитул от страницы
	\sethead{}{}{} % Содержание {слева}{по центру}{справа}
	% Нижний колонтитул
	\setfootrule{0cm} % Размер линии отделяющей колонтитул от страницы
	\setfoot{}{\thepage}{} % Содержание {слева}{по центру}{справа}
}
\pagestyle{main}

%My commands
\newcommand{\onefig}[4]
{\begin{figure}[H]
		\centering
		\includegraphics[width=#1\linewidth]{img/#2}
		\caption{\centering{#3}}
		#4
\end{figure}}

\newcommand{\twofig}[6]{\begin{figure}[H]
		\begin{minipage}[h]{0.49\linewidth}
			\centering{\includegraphics[width=#1\linewidth]{img/#2} \\ а)}
		\end{minipage}
		\hfill
		\begin{minipage}[h]{0.49\linewidth}
			\centering{\includegraphics[width=#3\linewidth]{img/#4} \\ б)}
		\end{minipage}
		\caption{\centering{#5}}
		#6
\end{figure}}

\newcommand{\threefig}[8]{\begin{figure}[H]
		\begin{minipage}[h]{0.49\linewidth}
			\centering{\includegraphics[width=#1\linewidth]{img/#2} \\ а)}
		\end{minipage}
		\hfill
		\begin{minipage}[h]{0.49\linewidth}
			\centering{\includegraphics[width=#3\linewidth]{img/#4} \\ б)}
		\end{minipage}
		\centering{\includegraphics[width=#5\linewidth]{img/#6} \\ в)}
		\caption{\centering{#7}}
		#8
\end{figure}}

\input{preamble/customCommand.tex}